% Complete documentation on the extended LaTeX markup used for Python
% documentation is available in ``Documenting Python'', which is part
% of the standard documentation for Python.  It may be found online
% at:
%
%     http://www.python.org/doc/current/doc/doc.html

\documentclass{manual}

\title{httpy}

\author{Chad W. L. Whitacre}

% Please at least include a long-lived email address;
% the rest is at your discretion.
\authoraddress{
	Zeta Design \&\ Development \\
	\url{http://www.zetadev.com/software/mode.py/} \\
	Email: \email{\ulink{chad@zetaweb.com}{mailto:chad@zetaweb.com}}
}

%\date{January 1, 1970} % update before release!
\date\today
				% Use an explicit date so that reformatting
				% doesn't cause a new date to be used.  Setting
				% the date to \today can be used during draft
				% stages to make it easier to handle versions.

\release{~~VERSION~~}			% release version; this is used to define the
				% \version macro

%\makeindex			% tell \index to actually write the .idx file
%\makemodindex			% If this contains a lot of module sections.


\begin{document}

\maketitle

\begin{abstract}

\noindent
\module{httpy} is a Python module that smooths out WSGI's most glaring warts.

\end{abstract}


\chapter{API Reference \label{api}}

The request side of WSGI---the "commons" of the \var{environ} mapping---is quite
nice. It honors the tradition of CGI, and it's just a mapping. Simple.

The response-side API is a little stiffer, because WSGI has to support edge
cases like serving large files, complex exception handling, and HTTP/1.1
features. This results in warts like \code{start_response}, and the requirement
that apps return an iterable. The intention in \ulink{PEP
333}{http://www.python.org/dev/peps/pep-0333/} is that these warts be smoothed
over at other layers; this is such a layer.

\module{httpy} provides the following classes:

\begin{classdesc}{Responder}{\var{app}}
Constructs a new \class{Responder} object. \var{app} is an extended WSGI
application: it may alternately return a string, or return or raise a
\class{Response} object.
\end{classdesc}

\begin{classdesc}{Response}{\optional{code} \optional{, body} \optional{,
    headers}}
Constructs a new \class{Response} object. If given, \var{code} must be an
integer; the default is
\ulink{200}{http://www.w3.org/Protocols/rfc2616/rfc2616-sec10.html#sec10.2.1}
(see \ulink{the HTTP
spec}{http://www.w3.org/Protocols/rfc2616/rfc2616-sec10.html} for other values
that will be meaningful to most HTTP clients). \var{body} must be a string.
\var{headers} may be a dictionary or a list of 2-tuples. \var{body} is second
rather than \var{headers} because one more often wants to specify a body without
headers than vice versa.
\end{classdesc}


\section{\class{Responder} Objects \label{responder}}

Instances of \class{httpy.Responder} are middleware that speak plain WSGI on the
server side, but they also accept strings and \class{Response} objects from the
application side. When a \class{Responder} receives a \class{Response} object,
that becomes the WSGI endpoint. When a \class{Responder} receives a string, it
creates a \class{Response} object with the string as the \var{body}, and the
\mailheader{Content-Type} set to \code{text/html}. This implicit
\class{Response} then becomes the endpoint.


\section{\class{Response} Objects \label{response}}

Instances of \class{httpy.Response} are also plain WSGI applications, with the
following data attributes. Note that values are only validated in the
constructor, so it is currently possible to return/raise a malformed
\class{Response} by setting instance attributes post-instantiation.

\begin{datadesc}{code}
The \ulink{HTTP code}{http://www.w3.org/Protocols/rfc2616/rfc2616-sec10.html} as an integer.
\end{datadesc}

\begin{datadesc}{body}
The message body as a string.
\end{datadesc}

\begin{datadesc}{headers}
The message headers as an instance of the standard library's
\ulink{\module{email.Message.Message}}{http://docs.python.org/lib/module-email.message.html}.
\end{datadesc}

When called, \class{Response} instances call \code{start_response} with
adaptations of \var{code} and \var{headers}, and return a one-item list
containing \var{body}.


\chapter{Example \label{example}}

Here is an example:

\begin{verbatim}
Python 2.5 (r25:52005, Sep 25 2006, 21:37:36)
[GCC 3.4.4 [FreeBSD] 20050518] on freebsd6
Type "help", "copyright", "credits" or "license" for more information.
>>>
>>> import httpy
>>> app = lambda env, start: "Greetings, program!"
>>> app = httpy.Responder(app)
>>>
>>> from wsgiref.simple_server import make_server
>>> server = make_server('', 8080, app)
>>> server.serve_forever() # now hit http://localhost:8080/
>>>
192.168.1.100 - - [09/Nov/2006 23:52:45] "GET / HTTP/1.1" 200 19
\end{verbatim}

\chapter{Copyright and License \label{legal}}

Copyright (c) 2006, Chad Whitacre <chad@zetaweb.com>. All rights reserved.

Redistribution and use in source and binary forms, with or without modification,
are permitted provided that the following conditions are met:

\begin{itemize}

\item{Redistributions of source code must retain the above copyright notice,
this list of conditions and the following disclaimer.}

\item{Redistributions in binary form must reproduce the above copyright notice,
this list of conditions and the following disclaimer in the documentation and/or
other materials provided with the distribution.}

\item{Neither the name of Chad Whitacre nor the names of any other contributors
or copyright holders may be used to endorse or promote products derived from
this software without specific prior written permission.}

\end{itemize}

THIS SOFTWARE IS PROVIDED BY THE COPYRIGHT HOLDERS AND CONTRIBUTORS "AS IS" AND
ANY EXPRESS OR IMPLIED WARRANTIES, INCLUDING, BUT NOT LIMITED TO, THE IMPLIED
WARRANTIES OF MERCHANTABILITY AND FITNESS FOR A PARTICULAR PURPOSE ARE
DISCLAIMED. IN NO EVENT SHALL THE COPYRIGHT OWNER OR CONTRIBUTORS BE LIABLE FOR
ANY DIRECT, INDIRECT, INCIDENTAL, SPECIAL, EXEMPLARY, OR CONSEQUENTIAL DAMAGES
(INCLUDING, BUT NOT LIMITED TO, PROCUREMENT OF SUBSTITUTE GOODS OR SERVICES;
LOSS OF USE, DATA, OR PROFITS; OR BUSINESS INTERRUPTION) HOWEVER CAUSED AND ON
ANY THEORY OF LIABILITY, WHETHER IN CONTRACT, STRICT LIABILITY, OR TORT
(INCLUDING NEGLIGENCE OR OTHERWISE) ARISING IN ANY WAY OUT OF THE USE OF THIS
SOFTWARE, EVEN IF ADVISED OF THE POSSIBILITY OF SUCH DAMAGE.


\end{document}
