% Complete documentation on the extended LaTeX markup used for Python
% documentation is available in ``Documenting Python'', which is part
% of the standard documentation for Python.  It may be found online
% at:
%
%     http://www.python.org/doc/current/doc/doc.html

\documentclass{manual}

\title{mode.py}

\author{Chad W. L. Whitacre}

% Please at least include a long-lived email address;
% the rest is at your discretion.
\authoraddress{
	Zeta Design \&\ Development \\
	\url{http://www.zetadev.com/software/mode.py/} \\
	Email: \email{\ulink{chad@zetaweb.com}{mailto:chad@zetaweb.com}}
}

%\date{January 1, 1970} % update before release!
\date\today
				% Use an explicit date so that reformatting
				% doesn't cause a new date to be used.  Setting
				% the date to \today can be used during draft
				% stages to make it easier to handle versions.

\release{1.0}			% release version; this is used to define the
				% \version macro

\makeindex			% tell \index to actually write the .idx file
\makemodindex			% If this contains a lot of module sections.


\begin{document}

\maketitle

\begin{abstract}

\noindent
mode.py is a Python module that models the life-cycle of an application as a
series of four modes.

\end{abstract}

\chapter{Introduction}

It is often valuable to maintain a distinction between various phases of an
application's lifecycle. This module calls these phases \dfn{modes}, and
identifies four of them, given here in conceptual life-cycle order:

\begin{tableii}{l|l}{code}{Mode}{Description}
\lineii{debugging}{The application is being actively debugged; exceptions may
    trigger an interactive debugger.}
\lineii{development}{The application is being actively developed; however,
    exceptions should not trigger interactive debugging.}
\lineii{staging}{The application is deployed in a mock-production
    environment.}
\lineii{production}{The application is in live use by its end users.}
\end{tableii}


The expectation is that various aspects of the application---logging, exception
handling, data sourcing---will adapt to the current mode. The mode is set in the
\envvar{PYTHONMODE} environment variable. This module provides API for
interacting with this variable. If \envvar{PYTHONMODE} is unset, it will be set
to development when this module is imported.

\chapter{API}

\section{Functions}

\begin{funcdesc}{get}{}
Return the current \envvar{PYTHONMODE} setting as a lowercase string; will raise
\exception  {EnvironmentError} if the (case-insensitive) setting is not one of
\code{debugging}, \code{development}, \code{staging}, or \code{production}.
\end{funcdesc}

\begin{funcdesc}{set}{mode}
Given a mode, set the PYTHONMODE environment variable and refresh the module\'s
boolean members. If given a bad mode, \exception{ValueError} is raised.
\end{funcdesc}

\begin{funcdesc}{setAPI}{}
Refresh the module's boolean members. Call this if you ever change
\envvar{PYTHONPATH} directly in the \code{os.environ} mapping.
\end{funcdesc}

\section{Members}

The module defines a number of boolean attributes reflecting the current mode
setting, including abbreviations and combinations. Uppercase versions of each of
the following are also defined (e.g., \code{DEBUGGING}).

\begin{datadesc}{debugging, deb}
\class{True} if \envvar{PYTHONMODE} is set to \code{debugging}.
\end{datadesc}
\begin{datadesc}{development, dev}
\class{True} if \envvar{PYTHONMODE} is set to \code{development}.
\end{datadesc}
\begin{datadesc}{staging, st}
\class{True} if \envvar{PYTHONMODE} is set to \code{staging}.
\end{datadesc}
\begin{datadesc}{production, prod}
\class{True} if \envvar{PYTHONMODE} is set to \code{production}.
\end{datadesc}
\begin{datadesc}{debugging_or_development, debdev, devdeb}
\class{True} if \envvar{PYTHONMODE} is set to \code{debugging} or \code{development}.
\end{datadesc}
\begin{datadesc}{staging_or_production, stprod}
\class{True} if \envvar{PYTHONMODE} is set to \code{staging} or \code{production}.
\end{datadesc}


\chapter{Example}

Example usage:

\begin{verbatim}
>>> import mode
>>> mode.set('development')     # can set the mode at runtime
>>> mode.get()                  # and access the current mode
'development'
>>> mode.development            # module defines boolean constants
True
>>> mode.PRODUCTION             # uppercase versions are also defined
False
>>> mode.dev                    # as are abbreviations
True
>>> mode.DEBDEV, mode.stprod    # and combinations
(True, False)
\end{verbatim}



% These are adding <Image> things to the last page, and I haven't cleaned up the
% inputs for indexing anyway.
%%
%%  The ugly "%begin{latexonly}" pseudo-environments are really just to
%%  keep LaTeX2HTML quiet during the \renewcommand{} macros; they're
%%  not really valuable.
%%
%%  If you don't want the Module Index, you can remove all of this up
%%  until the second \input line.
%%
%%begin{latexonly}
%\renewcommand{\indexname}{Module Index}
%%end{latexonly}
%\input{mod\jobname.ind}		% Module Index
%
%%begin{latexonly}
%\renewcommand{\indexname}{Index}
%%end{latexonly}
%\input{\jobname.ind}			% Index

\end{document}
