% Complete documentation on the extended LaTeX markup used for Python
% documentation is available in ``Documenting Python'', which is part
% of the standard documentation for Python.  It may be found online
% at:
%
%     http://www.python.org/doc/current/doc/doc.html

\documentclass{manual}

\title{httpy}

\author{Chad W. L. Whitacre}

% Please at least include a long-lived email address;
% the rest is at your discretion.
\authoraddress{
	Zeta Design \&\ Development \\
	\url{http://www.zetadev.com/software/httpy/} \\
	Email: \email{\ulink{chad@zetaweb.com}{mailto:chad@zetaweb.com}}
}

%\date{January 1, 1970} % update before release!
\date\today
				% Use an explicit date so that reformatting
				% doesn't cause a new date to be used.  Setting
				% the date to \today can be used during draft
				% stages to make it easier to handle versions.

\release{0.9}			% release version; this is used to define the
				% \version macro

\makeindex			% tell \index to actually write the .idx file
\makemodindex			% If this contains a lot of module sections.


\begin{document}

\maketitle

\begin{abstract}

\noindent
httpy is a sane and robust HTTP server and library for Python. With a
straightforward API and batteries included, it provides a stable and satisfying
HTTP bridge for your Python websites and web applications.

\end{abstract}

%\input{introduction}



% These are adding <Image> things to the last page, and I haven't cleaned up the
% inputs for indexing anyway.
%%
%%  The ugly "%begin{latexonly}" pseudo-environments are really just to
%%  keep LaTeX2HTML quiet during the \renewcommand{} macros; they're
%%  not really valuable.
%%
%%  If you don't want the Module Index, you can remove all of this up
%%  until the second \input line.
%%
%%begin{latexonly}
%\renewcommand{\indexname}{Module Index}
%%end{latexonly}
%\input{mod\jobname.ind}		% Module Index
%
%%begin{latexonly}
%\renewcommand{\indexname}{Index}
%%end{latexonly}
%\input{\jobname.ind}			% Index

\end{document}
